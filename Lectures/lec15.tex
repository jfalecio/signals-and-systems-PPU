\documentclass[11pt]{article}

\usepackage{amsmath,amssymb,mathtools}
\usepackage[margin=1in]{geometry}
\usepackage{enumitem}
\usepackage{xcolor}
\usepackage{microtype}
\usepackage{graphicx}
\usepackage{tikz,float}
\usepackage{subcaption}
\usepackage{amsthm}
\usepackage{hyperref}
\usepackage{array}
\usepackage{pgfplots}

\usetikzlibrary{shapes.geometric, arrows.meta, positioning, calc, decorations.markings}
\tikzset{
	block/.style={rectangle, draw, text width=6em, text centered, rounded corners, minimum height=10mm},
	sum/.style={circle, draw, node distance=1.5cm},
	line/.style={draw, -{Stealth[length=2.5mm, width=1.5mm]}}
}

\usepgfplotslibrary{groupplots}
\pgfplotsset{compat=1.18}

\pgfplotsset{
	myaxes/.style={
		axis lines=middle,
		axis line style={-latex},
		grid=major,
		grid style={gray!15},
		minor grid style={gray!35},
		xlabel style={at={(ticklabel* cs:1)}, anchor=north west},
		ylabel style={at={(ticklabel* cs:1)}, anchor=south east},
		every axis plot/.append style={thick}
	},
	myplotstyle/.style={
		width=14cm,
		height=7cm,
		axis lines=middle,
		axis line style={-Stealth},
		grid=both,
		minor tick num=1,
		major grid style={draw=gray!30},
		minor grid style={draw=gray!15},
		tick label style={font=\small, fill=white, inner sep=1.5pt},
		xlabel={$t$},
		ylabel={$x(t)$},
		xlabel style={anchor=north east, font=\small},
		ylabel style={anchor=south east, font=\small},
		samples=401,
	}
}

\newtheoremstyle{mynote}
{6pt}      % Space above
{6pt}      % Space below
{}          % Body font (normal, not italic)
{}          % Indent amount
{\bfseries} % Theorem head font
{.}         % Punctuation after theorem head
{.5em}      % Space after theorem head
{}          % Theorem head spec
\theoremstyle{mynote}
\newtheorem{definition}{Definition}
\newtheorem{proposition}{Proposition}
\newtheorem{example}{Example}
\newtheorem{remark}{Remark}
\newtheorem{theorem}{Theorem}
\newtheorem{corollary}{Corollary}

\newcommand{\T}{\mathcal{T}}
\newcommand{\R}{\mathbb{R}}
\newcommand{\Z}{\mathbb{Z}}
\newcommand{\C}{\mathbb{C}}
\newcommand{\conv}{\ast}
\newcommand{\dt}{\,\dd t}
\newcommand{\dd}{\mathrm{d}}
\newcommand{\imp}{\delta}
\newcommand{\sinc}[1]{\frac{\sin(\pi #1)}{\pi #1}}


\DeclareMathOperator{\rect}{rect}
\DeclareMathOperator{\Ev}{Ev}
\DeclareMathOperator{\Od}{Od}
\DeclareMathOperator{\sgn}{sgn}
\DeclareMathOperator{\step}{u}
\DeclareMathOperator{\tri}{tri}


\begin{document}
	% Reset figure counter for this lecture
	\renewcommand{\thefigure}{15.\arabic{figure}}
	
	% --- TITLE BLOCK ---
	\thispagestyle{empty}
	\noindent
	\begin{tabular*}{\textwidth}{l @{\extracolsep{\fill}} r}
		\textbf{Signals and Systems} & \textbf{Lecture 15} \\
		\textit{Dr. Ghandi Manasra and Ahmed Rabei} & \textit{Fall 2025} \\
	\end{tabular*}
	\hrule
	\vspace{0.4cm}
	\begin{center}
		\Large\textbf{Lecture 15: Properties of the DTFT \& LTI System Analysis}
	\end{center}
	\vspace{0.4cm}
	
	\section*{Reference}
	Oppenheim \& Willsky, \textit{Signals and Systems}, Chapter 5, Sections 5.3, 5.4, \& 5.8
	
	\section*{Review of Lecture 14}
	\begin{itemize}[noitemsep]
		\item The DTFT pair consists of an infinite sum (analysis) and an integral over a $2\pi$ interval (synthesis)
		\item The DTFT spectrum, $X(e^{j\omega})$, is \textbf{always a continuous and periodic function of $\omega$ with period $2\pi$}
		\item Examples: decaying exponential, rectangular pulse, unit impulse
		\item Convergence conditions and symmetry properties
	\end{itemize}
	
	\section*{15.1 Introduction}
	
	Having developed the Discrete-Time Fourier Transform (DTFT) in our last lecture, we now build the DTFT properties and immediately apply them to analyze discrete-time LTI systems described by linear constant-coefficient \textbf{difference equations}.
	
	The properties of the DTFT are direct counterparts to those of the CTFT, with one crucial difference: any frequency shifts must be interpreted modulo $2\pi$ due to the periodic nature of the DTFT.
	
	We use the notation $x[n] \stackrel{\mathcal{F}}{\longleftrightarrow} X(e^{j\omega})$ to indicate DTFT pairs throughout this lecture.
	
	\section*{15.2 Properties of the DTFT}
	\subsection*{15.2.1 Linearity Property}
	
	\begin{theorem}[Linearity]
		If $x_1[n] \stackrel{\mathcal{F}}{\longleftrightarrow} X_1(e^{j\omega})$ and $x_2[n] \stackrel{\mathcal{F}}{\longleftrightarrow} X_2(e^{j\omega})$, then:
		\[
		ax_1[n] + bx_2[n] \stackrel{\mathcal{F}}{\longleftrightarrow} aX_1(e^{j\omega}) + bX_2(e^{j\omega})
		\]
		for any constants $a, b \in \mathbb{C}$.
	\end{theorem}
	
	\textbf{Proof:} This follows directly from the linearity of the summation in the DTFT definition:
	\[
	\mathcal{F}\{ax_1[n] + bx_2[n]\} = \sum_{n=-\infty}^{\infty} (ax_1[n] + bx_2[n]) e^{-j\omega n} = a\sum_{n=-\infty}^{\infty} x_1[n] e^{-j\omega n} + b\sum_{n=-\infty}^{\infty} x_2[n] e^{-j\omega n} = aX_1(e^{j\omega}) + bX_2(e^{j\omega})
	\]
	
	\subsection*{15.2.2 Time Shifting Property}
	
	\begin{theorem}[Time Shifting]
		If $x[n] \stackrel{\mathcal{F}}{\longleftrightarrow} X(e^{j\omega})$, then:
		\[
		x[n - n_0] \stackrel{\mathcal{F}}{\longleftrightarrow} e^{-j\omega n_0}X(e^{j\omega})
		\]
		for any integer $n_0$.
	\end{theorem}
	
	\textbf{Proof:} Let $y[n] = x[n - n_0]$. Then:
	\[
	Y(e^{j\omega}) = \sum_{n=-\infty}^{\infty} x[n - n_0] e^{-j\omega n} = \sum_{m=-\infty}^{\infty} x[m] e^{-j\omega (m + n_0)} = e^{-j\omega n_0} \sum_{m=-\infty}^{\infty} x[m] e^{-j\omega m} = e^{-j\omega n_0} X(e^{j\omega})
	\]
	
	\textbf{Physical Interpretation:} Time shifting introduces a linear phase shift in the frequency domain. The magnitude spectrum remains unchanged, but the phase is modified by $-\omega n_0$.
	
	\subsection*{15.2.3 Frequency Shifting (Modulation) Property}
	
	\begin{theorem}[Frequency Shifting]
		If $x[n] \stackrel{\mathcal{F}}{\longleftrightarrow} X(e^{j\omega})$, then:
		\[
		e^{j\omega_0 n}x[n] \stackrel{\mathcal{F}}{\longleftrightarrow} X(e^{j(\omega - \omega_0)})
		\]
		for any real $\omega_0$.
	\end{theorem}
	
	\textbf{Proof:} 
	\[
	\mathcal{F}\{e^{j\omega_0 n}x[n]\} = \sum_{n=-\infty}^{\infty} e^{j\omega_0 n}x[n] e^{-j\omega n} = \sum_{n=-\infty}^{\infty} x[n] e^{-j(\omega - \omega_0)n} = X(e^{j(\omega - \omega_0)})
	\]
	
	\textbf{Physical Interpretation:} Multiplication by $e^{j\omega_0 n}$ shifts the spectrum by $\omega_0$ in frequency. Due to the $2\pi$ periodicity of the DTFT, this shift is interpreted modulo $2\pi$.
	
	\subsection*{15.2.4 Symmetry Properties for Real Signals}
	
	\textbf{Recall (from Lecture 14, Section 14.4.1):} If $x[n]$ is real, then $X(e^{-j\omega}) = X^*(e^{j\omega})$.
	
	\subsection*{15.2.5 First Difference Property}
	
	\begin{theorem}[First Difference]
		If $x[n] \stackrel{\mathcal{F}}{\longleftrightarrow} X(e^{j\omega})$, then:
		\[
		x[n] - x[n-1] \stackrel{\mathcal{F}}{\longleftrightarrow} (1 - e^{-j\omega})X(e^{j\omega})
		\]
	\end{theorem}
	
	\textbf{Proof:} Using linearity and time shifting:
	\[
	\mathcal{F}\{x[n] - x[n-1]\} = X(e^{j\omega}) - e^{-j\omega}X(e^{j\omega}) = (1 - e^{-j\omega})X(e^{j\omega})
	\]
	
	\textbf{Physical Interpretation:} The first difference is a high-pass operation. The factor $(1 - e^{-j\omega})$ is zero at $\omega = 0$ and largest at $\omega = \pi$, emphasizing high-frequency components.
	
	\subsection*{15.2.6 Accumulation (Summation) Property}
	
	\begin{theorem}[Accumulation]
		If $x[n] \stackrel{\mathcal{F}}{\longleftrightarrow} X(e^{j\omega})$, then:
		\[
		\sum_{k=-\infty}^{n} x[k] \stackrel{\mathcal{F}}{\longleftrightarrow} \frac{1}{1-e^{-j\omega}}X(e^{j\omega}) + \pi X(e^{j0})\sum_{k=-\infty}^{\infty} \delta(\omega - 2\pi k)
		\]
	\end{theorem}
	
	\textbf{Physical Interpretation:} Accumulation is the discrete-time analog of integration. The term $\frac{1}{1-e^{-j\omega}}$ provides the main frequency response, while the impulse train accounts for the DC component.
	
	\subsection*{15.2.7 Multiplication Property}
	
	\begin{theorem}[Multiplication Property]
		Multiplication in the time domain corresponds to \textbf{periodic convolution} in the frequency domain:
		\[
		x_1[n]x_2[n] \stackrel{\mathcal{F}}{\longleftrightarrow} \frac{1}{2\pi} \int_{2\pi} X_1(e^{j\theta})X_2(e^{j(\omega-\theta)})\,\dd\theta
		\]
	\end{theorem}
	
	The integral is over any interval of length $2\pi$.
	
	\subsection*{15.2.8 Parseval's Theorem}
	
	\begin{theorem}[Parseval's Theorem]
		For a signal $x[n]$ with DTFT $X(e^{j\omega})$:
		\[
		\sum_{n=-\infty}^{\infty} |x[n]|^2 = \frac{1}{2\pi} \int_{2\pi} |X(e^{j\omega})|^2 \dd\omega.
		\]
	\end{theorem}
	\begin{proof}
		\renewcommand{\qedsymbol}{}
		Start from the inverse DTFT:
		\[
		x[n] = \frac{1}{2\pi}\int_{2\pi} X(e^{j\omega})e^{j\omega n}\,\dd\omega.
		\]
		Then
		\[
		|x[n]|^2 = \frac{1}{(2\pi)^2}\int_{2\pi}\!\!\int_{2\pi} X(e^{j\omega})X^*(e^{j\theta}) e^{j(\omega-\theta)n}\,\dd\omega\,\dd\theta.
		\]
		Summing over all $n$ and using the orthogonality
		\[
		\sum_{n=-\infty}^{\infty} e^{j(\omega-\theta)n} = 2\pi\sum_{k=-\infty}^{\infty}\delta\big((\omega-\theta)-2\pi k\big),
		\]
		and the fact that the integrals are over any single $2\pi$-length interval, only the $k=0$ term contributes:
		\[
		\sum_{n=-\infty}^{\infty}|x[n]|^2
		= \frac{1}{(2\pi)^2}\int_{2\pi}\!\!\int_{2\pi} X(e^{j\omega})X^*(e^{j\theta}) \cdot 2\pi\,\delta(\omega-\theta)\,\dd\omega\,\dd\theta
		= \frac{1}{2\pi}\int_{2\pi} |X(e^{j\omega})|^2 \,\dd\omega.
		\]
	\end{proof}
	
	\subsection*{15.2.9 Frequency Differentiation Property}
	
	\begin{theorem}[Frequency Differentiation]
		Differentiation in the frequency domain corresponds to multiplication by $n$ in the time domain:
		\[
		nx[n] \stackrel{\mathcal{F}}{\longleftrightarrow} j\frac{dX(e^{j\omega})}{d\omega}
		\]
	\end{theorem}
	
	This property is useful for finding DTFTs of signals that are multiplied by $n$, such as ramp signals and other polynomial-weighted sequences.
	
	
	\section*{15.3 The Convolution Property}
	
	This is the most important property for analyzing discrete-time LTI systems. Just like in the continuous-time case, convolution in the time domain becomes simple multiplication in the frequency domain.
	
	\begin{theorem}[Convolution Property]
		\[ y[n] = x[n] * h[n] \stackrel{\mathcal{F}}{\longleftrightarrow} Y(e^{j\omega}) = X(e^{j\omega})H(e^{j\omega}) \]
	\end{theorem}
	
	The DTFT of the output of a discrete-time LTI system is simply the DTFT of the input multiplied by the system's \textbf{frequency response}, $H(e^{j\omega})$. The frequency response is the DTFT of the impulse response, $h[n]$.
	
	\begin{remark}
		 The difficult summation of \textbf{convolution} becomes simple point-by-point \textbf{multiplication} of continuous, periodic spectra.
	\end{remark}
	
	\section*{15.4 LTI Systems Described by Difference Equations}
	
	We can now use the convolution and time-shifting properties to analyze LTI systems described by linear constant-coefficient difference equations. The DTFT converts these recursive equations into simple algebraic equations.
	
	\subsection*{15.4.1 The Method}
	
	\begin{enumerate}[noitemsep]
		\item Start with the system's difference equation:
		$$\sum_{k=0}^{N} a_k y[n-k] = \sum_{k=0}^{M} b_k x[n-k]$$
		\item Take the DTFT of both sides of the equation.
		\item Apply the \textbf{Linearity} property to transform each term individually.
		\item Apply the \textbf{Time-Shifting} property: $\mathcal{F}\left\{y[n-k]\right\} = e^{-j\omega k} Y(e^{j\omega})$.
		\item This transforms the difference equation into an algebraic equation:
		$$\left(\sum_{k=0}^{N} a_k e^{-j\omega k}\right) Y(e^{j\omega}) = \left(\sum_{k=0}^{M} b_k e^{-j\omega k}\right) X(e^{j\omega})$$
		\item The frequency response of the system, $H(e^{j\omega})$, is the ratio $Y(e^{j\omega})/X(e^{j\omega})$:
		$$H(e^{j\omega}) = \frac{Y(e^{j\omega})}{X(e^{j\omega})} = \frac{\sum_{k=0}^{M} b_k e^{-j\omega k}}{\sum_{k=0}^{N} a_k e^{-j\omega k}}$$
	\end{enumerate}
	
	This systematic approach gives us a direct way to find the frequency response from the system's difference equation, enabling us to analyze the system's filtering characteristics and stability properties.
	
	\subsection*{15.4.2 Example: First-Order Recursive Lowpass Filter}
	
	\begin{example}
		\textbf{Difference Equation:} $y[n] - ay[n-1] = x[n]$
		
		\textbf{Take DTFT:}
		$$Y(e^{j\omega}) - a e^{-j\omega}Y(e^{j\omega}) = X(e^{j\omega})$$
		
		\textbf{Factor out $Y(e^{j\omega})$:}
		$$(1 - ae^{-j\omega})Y(e^{j\omega}) = X(e^{j\omega})$$
		
		\textbf{Solve for the Frequency Response:}
		$$H(e^{j\omega}) = \frac{Y(e^{j\omega})}{X(e^{j\omega})} = \frac{1}{1-ae^{-j\omega}}$$
		
		This is the same result we found for the decaying exponential impulse response $h[n]=a^n u[n]$.
	\end{example}
	
	\section*{15.5 Examples and Applications}
	
	\subsection*{15.5.1 Example: First-Order System Analysis}
	
	\begin{example}
		Consider the difference equation: $y[n] - \frac{1}{2}y[n-1] = x[n]$
		
		\textbf{Step 1: Take DTFT of both sides}
		\[
		Y(e^{j\omega}) - \frac{1}{2}e^{-j\omega}Y(e^{j\omega}) = X(e^{j\omega})
		\]
		
		\textbf{Step 2: Factor out $Y(e^{j\omega})$}
		\[
		\left(1 - \frac{1}{2}e^{-j\omega}\right)Y(e^{j\omega}) = X(e^{j\omega})
		\]
		
		\textbf{Step 3: Solve for the transfer function}
		\[
		H(e^{j\omega}) = \frac{Y(e^{j\omega})}{X(e^{j\omega})} = \frac{1}{1 - \frac{1}{2}e^{-j\omega}}
		\]
		
		\textbf{Step 4: Find the magnitude response}
		\[
		|H(e^{j\omega})| = \frac{1}{|1 - \frac{1}{2}e^{-j\omega}|} = \frac{1}{\sqrt{1 + \frac{1}{4} - \cos(\omega)}} = \frac{1}{\sqrt{\frac{5}{4} - \cos(\omega)}}
		\]
		
		\textbf{Step 5: Analyze the frequency response}
		\begin{itemize}[noitemsep]
			\item At $\omega = 0$: $|H(e^{j0})| = \frac{1}{\sqrt{\frac{5}{4} - 1}} = \frac{1}{\sqrt{\frac{1}{4}}} = 2$ (DC gain)
			\item At $\omega = \pi$: $|H(e^{j\pi})| = \frac{1}{\sqrt{\frac{5}{4} - (-1)}} = \frac{1}{\sqrt{\frac{9}{4}}} = \frac{2}{3}$ (High-frequency gain)
		\end{itemize}
		
		This system acts as a \textbf{lowpass filter} with a DC gain of 2 and high-frequency attenuation. The frequency response shows how the system preferentially passes low-frequency components while attenuating high-frequency components.
	\end{example}
	

	
	\subsection*{15.5.2 Example: System Interconnection}
	
	When LTI systems are connected in series (cascade), their overall transfer function is the product of individual transfer functions:
	\[
	H_{total}(e^{j\omega}) = H_1(e^{j\omega}) \cdot H_2(e^{j\omega}) \cdot \ldots \cdot H_n(e^{j\omega})
	\]
	
	For parallel connections, transfer functions add:
	\[
	H_{total}(e^{j\omega}) = H_1(e^{j\omega}) + H_2(e^{j\omega}) + \ldots + H_n(e^{j\omega})
	\]
	
	
	\section*{15.6 Practice Problems and Examples}
	
	\subsection*{15.6.1 Solved Example}
	
	\begin{example}
		Find the DTFT of $x[n] = n a^n u[n]$ for $|a| < 1$.
		
		\textbf{Solution:} We know that $a^n u[n] \stackrel{\mathcal{F}}{\longleftrightarrow} \frac{1}{1-ae^{-j\omega}}$.
		
		Using the differentiation property in frequency:
		\[
		n a^n u[n] \stackrel{\mathcal{F}}{\longleftrightarrow} j\frac{d}{d\omega}\left(\frac{1}{1-ae^{-j\omega}}\right) = \frac{ae^{-j\omega}}{(1-ae^{-j\omega})^2}
		\]
	\end{example}
	
	\subsection*{15.6.2 Practice Exercises}
	
	\begin{enumerate}[noitemsep]
		\item \textbf{Find the DTFT of $x[n] = \delta[n-n_0]$.}
		
		\textbf{Solution:} Using the definition of DTFT:
		\[
		X(e^{j\omega}) = \sum_{n=-\infty}^{\infty} \delta[n-n_0] e^{-j\omega n} = e^{-j\omega n_0}
		\]
		Therefore: $\delta[n-n_0] \stackrel{\mathcal{F}}{\longleftrightarrow} e^{-j\omega n_0}$
		
		\item \textbf{Show that if $x[n]$ is real and even, then $X(e^{j\omega})$ is real and even.}
		
		\textbf{Solution:} If $x[n]$ is real and even, then $x[n] = x[-n]$. Taking the DTFT:
		\[
		X(e^{j\omega}) = \sum_{n=-\infty}^{\infty} x[n] e^{-j\omega n} = \sum_{n=-\infty}^{\infty} x[-n] e^{-j\omega n}
		\]
		Let $m = -n$:
		\[
		X(e^{j\omega}) = \sum_{m=-\infty}^{\infty} x[m] e^{j\omega m} = \sum_{m=-\infty}^{\infty} x[m] e^{-j(-\omega) m} = X(e^{-j\omega})
		\]
		Since $X(e^{j\omega}) = X(e^{-j\omega})$, the DTFT is even. Also, for real $x[n]$, we have conjugate symmetry $X(e^{-j\omega}) = X^*(e^{j\omega})$. Combining both gives
		\[
		X(e^{j\omega}) = X(e^{-j\omega}) = X^*(e^{j\omega}) \;\Rightarrow\; X(e^{j\omega}) \text{ is real}.
		\]
		
		\item \textbf{Find the frequency response of the system $y[n] = x[n] - x[n-1]$.}
		
		\textbf{Solution:} Taking the DTFT of both sides:
		\[
		Y(e^{j\omega}) = X(e^{j\omega}) - e^{-j\omega}X(e^{j\omega}) = (1 - e^{-j\omega})X(e^{j\omega})
		\]
		Therefore: $H(e^{j\omega}) = \frac{Y(e^{j\omega})}{X(e^{j\omega})} = 1 - e^{-j\omega}$
		
		\item \textbf{Using Parseval's relation, find the energy of $x[n] = a^n u[n]$ for $|a| < 1$.}
		
		\textbf{Solution:} We know that $X(e^{j\omega}) = \frac{1}{1-ae^{-j\omega}}$. Using Parseval's theorem:
		\[
		\sum_{n=0}^{\infty} |a|^{2n} = \frac{1}{2\pi} \int_{2\pi} \frac{1}{|1-ae^{-j\omega}|^2} \dd\omega
		\]
		The left side is a geometric series: $\sum_{n=0}^{\infty} |a|^{2n} = \frac{1}{1-|a|^2}$
		Therefore: $E = \frac{1}{1-|a|^2}$
		
		\item \textbf{Find the output of the system $y[n] - \frac{1}{4}y[n-2] = x[n]$ when the input is $x[n] = \cos(\frac{\pi n}{4})$.}
		
		\textbf{Solution:} First, find the frequency response:
		\[
		H(e^{j\omega}) = \frac{1}{1 - \frac{1}{4}e^{-j2\omega}}
		\]
		For the input $x[n] = \cos(\frac{\pi n}{4})$, we have $\omega_0 = \frac{\pi}{4}$:
		\[
		H(e^{j\pi/4}) = \frac{1}{1 - \frac{1}{4}e^{-j\pi/2}} = \frac{1}{1 - \frac{1}{4}(-j)} = \frac{1}{1 + \frac{j}{4}}
		\]
		The output is: $y[n] = |H(e^{j\pi/4})|\cos(\frac{\pi n}{4} + \arg(H(e^{j\pi/4})))$
	\end{enumerate}
	
\section*{15.7 Summary and Next Lecture}
	\begin{itemize}[noitemsep]
		\item Used DTFT properties by recall (linearity, shifts, symmetry, difference/sum, multiplication, Parseval, frequency differentiation)
		\item Convolution in time $\leftrightarrow$ multiplication in frequency
		\item Difference equations turn into algebraic equations in the frequency domain
		\item Worked examples: first-order systems and interconnections
		\item \textbf{Next time:} Duality in the CTFT and CTFS/DTFT connections; using transform tables efficiently
	\end{itemize}
	
\end{document}
