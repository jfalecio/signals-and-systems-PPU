\documentclass[11pt]{article}

\usepackage{amsmath,amssymb,mathtools}
\usepackage[margin=1in]{geometry}
\usepackage{enumitem}
\usepackage{xcolor}
\usepackage{microtype}
\usepackage{graphicx}
\usepackage{tikz,float}
\usepackage{subcaption}
\usepackage{amsthm}
\usepackage{hyperref}
\usepackage{array}
\usepackage{pgfplots}

\usetikzlibrary{shapes.geometric, arrows.meta, positioning, calc, decorations.markings}
\tikzset{
	block/.style={rectangle, draw, text width=6em, text centered, rounded corners, minimum height=10mm},
	sum/.style={circle, draw, node distance=1.5cm},
	line/.style={draw, -{Stealth[length=2.5mm, width=1.5mm]}}
}

\usepgfplotslibrary{groupplots}
\pgfplotsset{compat=1.18}

\pgfplotsset{
	myaxes/.style={
		axis lines=middle,
		axis line style={-latex},
		grid=major,
		grid style={gray!15},
		minor grid style={gray!35},
		xlabel style={at={(ticklabel* cs:1)}, anchor=north west},
		ylabel style={at={(ticklabel* cs:1)}, anchor=south east},
		every axis plot/.append style={thick}
	},
	myplotstyle/.style={
		width=14cm,
		height=7cm,
		axis lines=middle,
		axis line style={-Stealth},
		grid=both,
		minor tick num=1,
		major grid style={draw=gray!30},
		minor grid style={draw=gray!15},
		tick label style={font=\small, fill=white, inner sep=1.5pt},
		xlabel={$t$},
		ylabel={$x(t)$},
		xlabel style={anchor=north east, font=\small},
		ylabel style={anchor=south east, font=\small},
		samples=401,
	}
}

\newtheoremstyle{mynote}
{6pt}      % Space above
{6pt}      % Space below
{}          % Body font (normal, not italic)
{}          % Indent amount
{\bfseries} % Theorem head font
{.}         % Punctuation after theorem head
{.5em}      % Space after theorem head
{}          % Theorem head spec
\theoremstyle{mynote}
\newtheorem{definition}{Definition}
\newtheorem{proposition}{Proposition}
\newtheorem{example}{Example}
\newtheorem{remark}{Remark}
\newtheorem{theorem}{Theorem}
\newtheorem{corollary}{Corollary}

\newcommand{\T}{\mathcal{T}}
\newcommand{\R}{\mathbb{R}}
\newcommand{\Z}{\mathbb{Z}}
\newcommand{\C}{\mathbb{C}}
\newcommand{\conv}{\ast}
\newcommand{\dt}{\,\dd t}
\newcommand{\dd}{\mathrm{d}}
\newcommand{\imp}{\delta}
\newcommand{\sinc}[1]{\frac{\sin(\pi #1)}{\pi #1}}


\DeclareMathOperator{\rect}{rect}
\DeclareMathOperator{\Ev}{Ev}
\DeclareMathOperator{\Od}{Od}
\DeclareMathOperator{\sgn}{sgn}
\DeclareMathOperator{\step}{u}
\DeclareMathOperator{\tri}{tri}


\begin{document}
	% Reset figure counter for this lecture
	\renewcommand{\thefigure}{16.\arabic{figure}}
	% --- TITLE BLOCK ---
	\thispagestyle{empty}
	\noindent
	\begin{tabular*}{\textwidth}{l @{\extracolsep{\fill}} r}
		\textbf{Signals and Systems} & \textbf{Lecture 16} \\
		\textit{Dr. Ghandi Manasra and Ahmed Rabei} & \textit{Fall 2025} \\
	\end{tabular*}
	\hrule
	\vspace{0.4cm}
	\begin{center}
		\Large\textbf{Lecture 16: Duality and Fourier Transform Tables}
	\end{center}
	\vspace{0.4cm}
	
	\section*{Reference}
	Oppenheim \& Willsky, \textit{Signals and Systems}, Chapter 4, Sections 4.3.6, 4.6; Chapter 5, Sections 5.6, 5.7
	
	\section*{Review of Lecture 15}
	\begin{itemize}[noitemsep]
		\item Complete toolkit of DTFT properties with mathematical proofs and physical interpretations
		\item Convolution property: $x[n] * h[n] \leftrightarrow X(e^{j\omega})H(e^{j\omega})$
		\item System analysis: DTFT converts difference equations into algebraic equations
		\item Practical implementation examples and problem-solving skills
		\item Periodicity awareness: $2\pi$ periodicity affects all DTFT properties
	\end{itemize}
	
\section*{16.1 Introduction}
	
	We will explore one of the most important properties of Fourier analysis: \textbf{duality}. We then focus on the practical skill of using tables of transform pairs and properties to solve problems efficiently.
	\\
	The duality property shows that the roles of time and frequency can be interchanged in Fourier transform pairs.
	
\subsection*{16.1.1 Key Function Definitions}
	
	Before we begin, let's establish clear definitions for the fundamental functions we'll encounter:
	
	\begin{definition}[Normalized Sinc Function]
		The normalized sinc function is defined as:
		\[
		\sinc{x} = \frac{\sin(\pi x)}{\pi x}
		\]
		with the special case $\sinc{0} = 1$.
	\end{definition}
	
	\begin{definition}[Rectangular Function]
		The rectangular function is defined as:
		\[
		\rect(t) = \begin{cases} 1, & |t| < \frac{1}{2} \\ 0, & |t| > \frac{1}{2} \end{cases}
		\]
	\end{definition}
\newpage	
\section*{16.2 The Duality Property of the CTFT}
	
	The duality property arises from the remarkable similarity in the structure of the forward and inverse Fourier transform integrals. 
		
	\subsection*{16.2.1 The Duality Property}
	
	\begin{theorem}[Duality Property]
		If $x(t) \stackrel{\mathcal{F}}{\longleftrightarrow} X(j\omega)$, then:
		\[
		X(t) \stackrel{\mathcal{F}}{\longleftrightarrow} 2\pi x(-\omega)
		\]
	\end{theorem}
	
	\textbf{Intuition:} If a time-domain signal $x(t)$ has a Fourier transform $X(j\omega)$, then a new time-domain signal that has the \textit{same functional form} as the spectrum $X(j\omega)$ will have a spectrum that has the \textit{same functional form} as the original time-domain signal $x(t)$, with a scaling factor and a time reversal.
	
	\subsection*{16.2.2 Rectangular Pulse and Sinc Function}
	
	\begin{example}
		\textbf{Step 1:} Start with a known pair
		
		Consider a rectangular pulse in time:
		\[
		x(t) = \rect\left(\frac{t}{2T_1}\right) = \begin{cases} 1, & |t| < T_1 \\ 0, & |t| > T_1 \end{cases}
		\]
		
		Its Fourier transform is:
		\[
		X(j\omega) = \frac{2\sin(\omega T_1)}{\omega} = 2T_1 \sinc{\frac{\omega T_1}{\pi}}
		\]
		
		\textbf{Step 2:} Apply Duality
		
		Now consider a time-domain signal that is shaped like a sinc function:
		\[
		y(t) = \frac{\sin(Wt)}{\pi t} = \frac{W}{\pi} \sinc{\frac{Wt}{\pi}}
		\]
		
		\textbf{Step 3:} Recognize the shapes
		
		We can see that $y(t)$ has the same functional form as $X(j\omega)$. Duality tells us that its spectrum, $Y(j\omega)$, must have the same functional form as the original rectangular pulse, $x(t)$.
		
		\textbf{Step 4:} Result
		\[
		\frac{\sin(Wt)}{\pi t} \stackrel{\mathcal{F}}{\longleftrightarrow} Y(j\omega) = \rect\left(\frac{\omega}{2W}\right) = \begin{cases} 1, & |\omega| < W \\ 0, & |\omega| > W \end{cases}
		\]
		
		This shows that the impulse response of an ideal lowpass filter is a sinc function.
	\end{example}
	
	\begin{remark}
		Duality can be a confusing concept initially. The key is to think about the \textit{shapes} of the functions. If shape A in time gives shape B in frequency, then shape B in time gives shape A in frequency (with some scaling and reflection).
	\end{remark}
	
	\begin{example}[Find the Fourier transform of $x(t) = \frac{1}{1 + t^2}$]
		We can use duality to solve this problem efficiently.
		
		\textbf{Step 1:} We know that $e^{-a|t|} \stackrel{\mathcal{F}}{\longleftrightarrow} \frac{2a}{a^2 + \omega^2}$ for $a > 0$.
		
		\textbf{Step 2:} Apply duality. If $e^{-a|t|} \leftrightarrow \frac{2a}{a^2 + \omega^2}$, then:
		\[
		\frac{2a}{a^2 + t^2} \stackrel{\mathcal{F}}{\longleftrightarrow} 2\pi e^{-a|\omega|}
		\]
		
		\textbf{Step 3:} For $a = 1$:
		\[
		\frac{2}{1 + t^2} \stackrel{\mathcal{F}}{\longleftrightarrow} 2\pi e^{-|\omega|}
		\]
		
		\textbf{Step 4:} Therefore:
		\[
		\frac{1}{1 + t^2} \stackrel{\mathcal{F}}{\longleftrightarrow} \pi e^{-|\omega|}
		\]
	\end{example}

	\begin{example}[Find the Fourier transform of $x(t) = e^{-|t|}$ using duality]
		We know from the previous example that $\frac{1}{1 + t^2} \leftrightarrow \pi e^{-|\omega|}$.
		
		By duality: If $x(t) = \frac{1}{1 + t^2}$ and $X(j\omega) = \pi e^{-|\omega|}$, then the transform of $X(t) = \pi e^{-|t|}$ is $2\pi x(-\omega) = 2\pi \frac{1}{1 + (-\omega)^2} = \frac{2\pi}{1 + \omega^2}$.
		
		Therefore: $\pi e^{-|t|} \leftrightarrow \frac{2\pi}{1 + \omega^2}$
		
		By linearity (dividing by $\pi$): $e^{-|t|} \leftrightarrow \frac{2}{1 + \omega^2}$
	\end{example}

\section*{16.3 Using CTFT Properties and Pairs}
	
	Now we will see how to solve problems by combining properties and pairs from tables. This systematic approach allows us to find Fourier transforms efficiently without resorting to integration.
	
	\subsection*{16.3.1 CTFT Properties Table}
	
	\begin{table}[H]
		\centering
		\caption{Properties of the Continuous-Time Fourier Transform}
		\small
		\renewcommand{\arraystretch}{1.3}
		\resizebox{1\textwidth}{!}{%
		\begin{tabular}{|>{\raggedright\arraybackslash}p{3cm}|>{\raggedright\arraybackslash}p{4cm}|>{\raggedright\arraybackslash}p{6cm}|}
			\hline
			\textbf{Property} & \textbf{Aperiodic Signal} & \textbf{Fourier Transform} \\
			\hline
			& $x(t)$ & $X(j\omega)$ \\
			& $y(t)$ & $Y(j\omega)$ \\
			\hline
			Linearity & $ax(t)+by(t)$ & $aX(j\omega) + bY(j\omega)$ \\
			\hline
			Time Shifting & $x(t-t_0)$ & $e^{-j\omega t_0}X(j\omega)$ \\
			\hline
			Frequency Shifting & $e^{j\omega_0 t}x(t)$ & $X(j(\omega-\omega_0))$ \\
			\hline
			Conjugation & $x^*(t)$ & $X^*(-j\omega)$ \\
			\hline
			Time Reversal & $x(-t)$ & $X(-j\omega)$ \\
			\hline
			Time and Frequency Scaling & $x(at)$ & $\frac{1}{|a|}X(\frac{j\omega}{a})$ \\
			\hline
			Convolution & $x(t) * y(t)$ & $X(j\omega)Y(j\omega)$ \\
			\hline
			Multiplication & $x(t)y(t)$ & $\frac{1}{2\pi}\int_{-\infty}^{+\infty} X(j\theta)Y(j(\omega-\theta))\,\dd\theta$ \\
			\hline
			Differentiation in Time & $\frac{\dd}{\dd t}x(t)$ & $j\omega X(j\omega)$ \\
			\hline
			Integration & $\int_{-\infty}^t x(\tau)\,\dd\tau$ & $\frac{1}{j\omega}X(j\omega) + \pi X(0)\delta(\omega)$ \\
			\hline
			Differentiation in Frequency & $tx(t)$ & $j\frac{\dd}{\dd\omega}X(j\omega)$ \\
			\hline
			Conjugate Symmetry & $x(t)$ real & $X(j\omega) = X^*(-j\omega)$ \\
			for Real Signals & & $\text{Re}\{X(j\omega)\} = \text{Re}\{X(-j\omega)\}$ \\
			& & $\text{Im}\{X(j\omega)\} = -\text{Im}\{X(-j\omega)\}$ \\
			& & $|X(j\omega)| = |X(-j\omega)|$ \\
			& & $\angle X(j\omega) = -\angle X(-j\omega)$ \\
			\hline
			Symmetry for Real and Even Signals & $x(t)$ real and even & $X(j\omega)$ real and even \\
			\hline
			Symmetry for Real and Odd Signals & $x(t)$ real and odd & $X(j\omega)$ purely imaginary and odd \\
			\hline
			Even-Odd Decomposition & $x_e(t) = \Ev\{x(t)\}$ & $\text{Re}\{X(j\omega)\}$ \\
			of Real Signals & $x_o(t) = \Od\{x(t)\}$ & $j\text{Im}\{X(j\omega)\}$ \\
			\hline
			\multicolumn{3}{|c|}{\textbf{Parseval's Relation for Aperiodic Signals}} \\
			\hline
			\multicolumn{3}{|c|}{$\int_{-\infty}^{+\infty} |x(t)|^2 \,\dd t = \frac{1}{2\pi}\int_{-\infty}^{+\infty} |X(j\omega)|^2 \,\dd\omega$} \\
			\hline
		\end{tabular}%
		}
		\label{tab:ctft_properties}
	\end{table}
	
	\subsection*{16.3.2 CTFT Basic Transform Pairs}
	
	\begin{table}[H]
		\centering
		\caption{Basic Continuous-Time Fourier Transform Pairs}
		\small
		\renewcommand{\arraystretch}{1.3}
		\resizebox{1\textwidth}{!}{%
		\begin{tabular}{|>{\raggedright\arraybackslash}p{4cm}|>{\raggedright\arraybackslash}p{4cm}|>{\raggedright\arraybackslash}p{5cm}|}
			\hline
			\textbf{Signal} & \textbf{Fourier Transform} & \textbf{Fourier Series Coefficients (if periodic)} \\
			\hline
			$\sum_{k=-\infty}^{+\infty} a_k e^{jk\omega_0 t}$ & $2\pi\sum_{k=-\infty}^{+\infty} a_k \delta(\omega-k\omega_0)$ & $a_k$ \\
			\hline
			$e^{j\omega_0 t}$ & $2\pi\delta(\omega-\omega_0)$ & $a_1=1$, $a_k=0$, otherwise \\
			\hline
			$\cos(\omega_0 t)$ & $\pi[\delta(\omega-\omega_0) + \delta(\omega+\omega_0)]$ & $a_1=a_{-1}=\frac{1}{2}$, $a_k=0$, otherwise \\
			\hline
			$\sin(\omega_0 t)$ & $\frac{\pi}{j}[\delta(\omega-\omega_0) - \delta(\omega+\omega_0)]$ & $a_1=-a_{-1}=\frac{1}{2j}$, $a_k=0$, otherwise \\
			\hline
			$x(t)=1$ & $2\pi\delta(\omega)$ & $a_0=1$, $a_k=0, k \ne 0$ \\
			\hline
			Periodic square wave & $\sum_{k=-\infty}^{+\infty} \frac{2\sin(k\omega_0 T_1)}{k} \delta(\omega-k\omega_0)$ & $\frac{\sin(k\omega_0 T_1)}{k\pi} = \frac{2T_1}{T} \sinc{\frac{2kT_1}{T}}$ \\
			$x(t)=\begin{cases} 1, & |t|<T_1 \\ 0, & T_1<|t|\le T/2 \end{cases}$ & & \\
			and $x(t+T)=x(t)$ & & \\
			\hline
			$\sum_{n=-\infty}^{+\infty} \delta(t-nT)$ & $\frac{2\pi}{T}\sum_{k=-\infty}^{+\infty} \delta(\omega - \frac{2\pi k}{T})$ & $a_k=\frac{1}{T}$ for all k \\
			\hline
			$x(t)=\begin{cases} 1, & |t|<T_1 \\ 0, & |t|>T_1 \end{cases}$ & $\frac{2\sin(\omega T_1)}{\omega}$ & \\
			\hline
			$\frac{\sin(Wt)}{\pi t}$ & $X(j\omega)=\begin{cases} 1, & |\omega|<W \\ 0, & |\omega|>W \end{cases}$ & \\
			\hline
			$\delta(t)$ & 1 & \\
			\hline
			$u(t)$ & $\frac{1}{j\omega} + \pi\delta(\omega)$ & \\
			\hline
			$\delta(t-t_0)$ & $e^{-j\omega t_0}$ & \\
			\hline
			$e^{-at}u(t), \text{Re}\{a\}>0$ & $\frac{1}{a+j\omega}$ & \\
			\hline
			$te^{-at}u(t), \text{Re}\{a\}>0$ & $\frac{1}{(a+j\omega)^2}$ & \\
			\hline
			$\frac{t^{n-1}}{(n-1)!}e^{-at}u(t), \text{Re}\{a\}>0$ & $\frac{1}{(a+j\omega)^n}$ & \\
			\hline
		\end{tabular}%
		}
		\label{tab:ctft_pairs_comprehensive}
	\end{table}
	
	\subsection*{16.3.3 Examples Using Properties and Pairs}
	
	\begin{example}[Find the Fourier transform of $x(t) = 3\cos(2t - \pi/4)$]
		We'll use the systematic approach with tables.
		
		\textbf{Step 1:} Decompose the signal
		\[
		x(t) = 3\cos(2t - \pi/4) = 3\cos(2(t - \pi/8))
		\]
		
		\textbf{Step 2:} Identify basic form
		From the table: $\cos(\omega_0 t) \leftrightarrow \pi[\delta(\omega - \omega_0) + \delta(\omega + \omega_0)]$
		
		\textbf{Step 3:} Apply scaling
		\[
		3\cos(2t) \leftrightarrow 3\pi[\delta(\omega - 2) + \delta(\omega + 2)]
		\]
		
		\textbf{Step 4:} Apply time shifting
		\[
		3\cos(2(t - \pi/8)) \leftrightarrow 3\pi[\delta(\omega - 2) + \delta(\omega + 2)]e^{-j\omega\pi/8}
		\]
		
		\[
		X(j\omega) = 3\pi[\delta(\omega - 2) + \delta(\omega + 2)]e^{-j\omega\pi/8}
		\]
	\end{example}

	\begin{example}[Find the Fourier transform of $x(t) = \frac{d}{dt}\left(e^{-(t-2)}u(t-2)\right)$]
		We'll break this down systematically, combining time shifting, differentiation, and basic transform pairs.
		
		\textbf{Step 1:} Identify basic shape
		From the table: $e^{-at}u(t) \leftrightarrow \frac{1}{a + j\omega}$
		
		For $a = 1$: $e^{-t}u(t) \leftrightarrow \frac{1}{1 + j\omega}$
		
		\textbf{Step 2:} Apply time shifting
		\[
		e^{-(t-2)}u(t-2) \leftrightarrow \frac{e^{-j2\omega}}{1 + j\omega}
		\]
		
		\textbf{Step 3:} Apply differentiation property
		\[
		\frac{d}{dt}\left(e^{-(t-2)}u(t-2)\right) \leftrightarrow j\omega \cdot \frac{e^{-j2\omega}}{1 + j\omega}
		\]
		
		\[
		X(j\omega) = \frac{j\omega e^{-j2\omega}}{1 + j\omega}
		\]
	\end{example}

	\begin{example}[Use the convolution property to find the Fourier transform of $x(t) = \rect(t) * \rect(t)$]
		We know that $\rect\left(\frac{t}{2T_1}\right) \leftrightarrow 2T_1 \sinc{\frac{\omega T_1}{\pi}}$. For $\rect(t)$, we set $2T_1 = 1$, so $T_1 = 1/2$. Therefore:
		\[
		\rect(t) \leftrightarrow 2(1/2)\sinc{\frac{\omega(1/2)}{\pi}} = \sinc{\frac{\omega}{2\pi}}
		\]
		By convolution property: $\rect(t) * \rect(t) \leftrightarrow \left(\sinc{\frac{\omega}{2\pi}}\right)^2$
	\end{example}

	\begin{example}[Find the Fourier transform of $x(t) = t e^{-t}u(t)$ using the frequency differentiation property]
		We know that $e^{-t}u(t) \leftrightarrow \frac{1}{1 + j\omega}$.
		
		By frequency differentiation: $t e^{-t}u(t) \leftrightarrow j\frac{\dd}{\dd\omega}\left(\frac{1}{1 + j\omega}\right) = \frac{1}{(1 + j\omega)^2}$
	\end{example}

	\begin{example}[Find the Fourier transform of $x(t) = \rect(t/2)$ using tables]
		From our known transform pair, $\rect\left(\frac{t}{2T_1}\right) \leftrightarrow 2T_1 \sinc{\frac{\omega T_1}{\pi}}$. For the signal $x(t) = \rect(t/2)$, we set $2T_1 = 2$, which means $T_1 = 1$. Substituting this into the transform gives:
		\[
		\mathcal{F}\{\rect(t/2)\} = 2(1)\sinc{\frac{\omega(1)}{\pi}} = 2\sinc{\frac{\omega}{\pi}}
		\]
	\end{example}

	\begin{example}[Find the Fourier transform of $x(t) = \frac{\sin(t)}{t}$]
		We can start from the known Fourier transform pair for an ideal low-pass filter's impulse response:
		\[
		\frac{\sin(Wt)}{\pi t} \stackrel{\mathcal{F}}{\longleftrightarrow} \rect\left(\frac{\omega}{2W}\right) = \begin{cases} 1, & |\omega| < W \\ 0, & |\omega| > W \end{cases}
		\]
		Let $W=1$. The pair becomes:
		\[
		\frac{\sin(t)}{\pi t} \stackrel{\mathcal{F}}{\longleftrightarrow} \rect\left(\frac{\omega}{2}\right) = \begin{cases} 1, & |\omega| < 1 \\ 0, & |\omega| > 1 \end{cases}
		\]
		To find the transform of just $\frac{\sin(t)}{t}$, we simply multiply by $\pi$:
		\[
		\mathcal{F}\left\{\frac{\sin(t)}{t}\right\} = \pi \cdot \rect\left(\frac{\omega}{2}\right) = \begin{cases} \pi, & |\omega| < 1 \\ 0, & |\omega| > 1 \end{cases}
		\]
	\end{example}

\section*{16.6 Reference Tables}
	
	Just as we used the CTFT tables, corresponding tables exist for all Fourier representations. They are included here for your reference.
	
	\subsection*{16.6.1 Fourier Series Properties Tables}
	
	\begin{table}[H]
		\centering
		\caption{Properties of Continuous-Time Fourier Series}
		\small
		\renewcommand{\arraystretch}{1.3}
		\resizebox{1\textwidth}{!}{%
		\begin{tabular}{|>{\raggedright\arraybackslash}p{3cm}|>{\raggedright\arraybackslash}p{4cm}|>{\raggedright\arraybackslash}p{6cm}|}
			\hline
			\textbf{Property} & \textbf{Periodic Signal} & \textbf{Fourier Series Coefficients} \\
			\hline
			& $x(t)$ Periodic with period $T$ and & $a_k$ \\
			& $y(t)$ fundamental frequency $\omega_0=2\pi/T$ & $b_k$ \\
			\hline
			Linearity & $Ax(t) + By(t)$ & $Aa_k + Bb_k$ \\
			\hline
			Time Shifting & $x(t-t_0)$ & $a_k e^{-jk\omega_0 t_0}$ \\
			\hline
			Frequency Shifting & $e^{jM\omega_0 t}x(t)$ & $a_{k-M}$ \\
			\hline
			Conjugation & $x^*(t)$ & $a_{-k}^*$ \\
			\hline
			Time Reversal & $x(-t)$ & $a_{-k}$ \\
			\hline
			Time Scaling & $x(\alpha t), \alpha > 0$ & $a_k$ \\
			& (periodic with period $T/\alpha$) & \\
			\hline
			Periodic Convolution & $\int_T x(\tau)y(t-\tau)\,\dd\tau$ & $Ta_k b_k$ \\
			\hline
			Multiplication & $x(t)y(t)$ & $\sum_{l=-\infty}^{+\infty} a_l b_{k-l}$ \\
			\hline
			Differentiation & $\frac{dx(t)}{dt}$ & $jk\omega_0 a_k$ \\
			\hline
			Integration & $\int_{-\infty}^t x(\tau)\,\dd\tau$ & $(\frac{1}{jk\omega_0})a_k$ \\
			& (finite valued and periodic only if $a_0=0$) & \\
			\hline
			Conjugate Symmetry & $x(t)$ real & $a_k = a_{-k}^*$ \\
			for Real Signals & & $\text{Re}\{a_k\} = \text{Re}\{a_{-k}\}$ \\
			& & $\text{Im}\{a_k\} = -\text{Im}\{a_{-k}\}$ \\
			& & $|a_k| = |a_{-k}|$ \\
			& & $\angle a_k = -\angle a_{-k}$ \\
			\hline
			Real and Even Signals & $x(t)$ real and even & $a_k$ real and even \\
			\hline
			Real and Odd Signals & $x(t)$ real and odd & $a_k$ purely imaginary and odd \\
			\hline
			Even-Odd Decomposition & $x_e(t) = \Ev\{x(t)\}$ & $\text{Re}\{a_k\}$ \\
			of Real Signals & $x_o(t) = \Od\{x(t)\}$ & $j\text{Im}\{a_k\}$ \\
			\hline
			\multicolumn{3}{|c|}{\textbf{Parseval's Relation for Periodic Signals}} \\
			\hline
			\multicolumn{3}{|c|}{$\frac{1}{T}\int_T |x(t)|^2 \,\dd t = \sum_{k=-\infty}^{+\infty} |a_k|^2$} \\
			\hline
		\end{tabular}%
		}
		\label{tab:ctfs_properties}
	\end{table}
	
	\begin{table}[H]
		\centering
		\caption{Properties of Discrete-Time Fourier Series}
		\small
		\renewcommand{\arraystretch}{1.3}
		\resizebox{1\textwidth}{!}{%
		\begin{tabular}{|>{\raggedright\arraybackslash}p{3cm}|>{\raggedright\arraybackslash}p{4cm}|>{\raggedright\arraybackslash}p{6cm}|}
			\hline
			\textbf{Property} & \textbf{Periodic Signal} & \textbf{Fourier Series Coefficients} \\
			\hline
			& $x[n]$ Periodic with period $N$ and & $a_k$ Periodic with \\
			& $y[n]$ fundamental frequency $\omega_0=2\pi/N$ & $b_k$ period $N$ \\
			\hline
			Linearity & $Ax[n] + By[n]$ & $Aa_k + Bb_k$ \\
			\hline
			Time Shifting & $x[n-n_0]$ & $a_k e^{-jk(2\pi/N)n_0}$ \\
			\hline
			Frequency Shifting & $e^{jM(2\pi/N)n}x[n]$ & $a_{k-M}$ \\
			\hline
			Conjugation & $x^*[n]$ & $a_{-k}^*$ \\
			\hline
			Time Reversal & $x[-n]$ & $a_{-k}$ \\
			\hline
			Periodic Convolution & $\sum_{r=\langle N \rangle} x[r]y[n-r]$ & $Na_k b_k$ \\
			\hline
			Multiplication & $x[n]y[n]$ & $\sum_{l=\langle N \rangle} a_l b_{k-l}$ \\
			\hline
			First Difference & $x[n] - x[n-1]$ & $(1 - e^{-jk(2\pi/N)})a_k$ \\
			\hline
			Running Sum & $\sum_{k=-\infty}^{n} x[k]$ & $(\frac{1}{1-e^{-jk(2\pi/N)}})a_k$ \\
			& (finite valued and periodic & \\
			& only if $a_0=0$) & \\
			\hline
			Conjugate Symmetry & $x[n]$ real & $a_k = a_{-k}^*$ \\
			for Real Signals & & $\text{Re}\{a_k\} = \text{Re}\{a_{-k}\}$ \\
			& & $\text{Im}\{a_k\} = -\text{Im}\{a_{-k}\}$ \\
			& & $|a_k| = |a_{-k}|$ \\
			& & $\angle a_k = -\angle a_{-k}$ \\
			\hline
			Real and Even Signals & $x[n]$ real and even & $a_k$ real and even \\
			\hline
			Real and Odd Signals & $x[n]$ real and odd & $a_k$ purely imaginary and odd \\
			\hline
			Even-Odd Decomposition & $x_e[n] = \Ev\{x[n]\}$ & $\text{Re}\{a_k\}$ \\
			of Real Signals & $x_o[n] = \Od\{x[n]\}$ & $j\text{Im}\{a_k\}$ \\
			\hline
			\multicolumn{3}{|c|}{\textbf{Parseval's Relation for Periodic Signals}} \\
			\hline
			\multicolumn{3}{|c|}{$\frac{1}{N}\sum_{n=\langle N \rangle} |x[n]|^2 = \sum_{k=\langle N \rangle} |a_k|^2$} \\
			\hline
		\end{tabular}%
		}
		\label{tab:dtfs_properties}
	\end{table}
	
	\subsection*{16.6.2 DTFT Properties and Pairs}
	
	\begin{table}[H]
		\centering
		\small
		\renewcommand{\arraystretch}{1.3}
		\resizebox{1\textwidth}{!}{%
		\begin{tabular}{|>{\raggedright\arraybackslash}p{3cm}|>{\raggedright\arraybackslash}p{4cm}|>{\raggedright\arraybackslash}p{6cm}|}
			\hline
			\textbf{Property} & \textbf{Aperiodic Signal} & \textbf{Fourier Transform} \\
			\hline
			& $x[n]$ & $X(e^{j\omega})$ periodic with \\
			& $y[n]$ & $Y(e^{j\omega})$ period $2\pi$ \\
			\hline
			Linearity & $ax[n]+by[n]$ & $aX(e^{j\omega}) + bY(e^{j\omega})$ \\
			\hline
			Time Shifting & $x[n-n_0]$ & $e^{-j\omega n_0}X(e^{j\omega})$ \\
			\hline
			Frequency Shifting & $e^{j\omega_0 n}x[n]$ & $X(e^{j(\omega-\omega_0)})$ \\
			\hline
			Conjugation & $x^*[n]$ & $X^*(e^{-j\omega})$ \\
			\hline
			Time Reversal & $x[-n]$ & $X(e^{-j\omega})$ \\
			\hline
			Time Expansion & $x_{(k)}[n]=\begin{cases} x[n/k], & \text{if } n \text{ is a} \\ & \text{multiple of } k \\ 0, & \text{if } n \text{ is not a} \\ & \text{multiple of } k \end{cases}$ & $X(e^{jk\omega})$ \\
			\hline
			Convolution & $x[n]*y[n]$ & $X(e^{j\omega})Y(e^{j\omega})$ \\
			\hline
			Multiplication & $x[n]y[n]$ & $\frac{1}{2\pi}\int_{2\pi} X(e^{j\theta})Y(e^{j(\omega-\theta)})\,\dd\theta$ \\
			\hline
			Differencing in Time & $x[n]-x[n-1]$ & $(1-e^{-j\omega})X(e^{j\omega})$ \\
			\hline
			Accumulation & $\sum_{m=-\infty}^{n} x[m]$ & $\frac{1}{1-e^{-j\omega}}X(e^{j\omega}) +$ \\
			& & $\pi X(e^{j0})\sum_{k=-\infty}^{+\infty}\delta(\omega-2\pi k)$ \\
			\hline
			Differentiation in Frequency & $nx[n]$ & $j\frac{\dd X(e^{j\omega})}{\dd\omega}$ \\
			\hline
			Conjugate Symmetry & $x[n]$ real & $X(e^{j\omega}) = X^*(e^{-j\omega})$ \\
			for Real Signals & & $\text{Re}\{X(e^{j\omega})\} = \text{Re}\{X(e^{-j\omega})\}$ \\
			& & $\text{Im}\{X(e^{j\omega})\} = -\text{Im}\{X(e^{-j\omega})\}$ \\
			& & $|X(e^{j\omega})| = |X(e^{-j\omega})|$ \\
			& & $\angle X(e^{j\omega}) = -\angle X(e^{-j\omega})$ \\
			\hline
			Symmetry for Real, Even Signals & $x[n]$ real and even & $X(e^{j\omega})$ real and even \\
			\hline
			Symmetry for Real, Odd Signals & $x[n]$ real and odd & $X(e^{j\omega})$ purely imaginary and odd \\
			\hline
			Even-odd Decomposition & $x_e[n] = \Ev\{x[n]\}$ & $\text{Re}\{X(e^{j\omega})\}$ \\
			of Real Signals & $x_o[n] = \Od\{x[n]\}$ & $j\text{Im}\{X(e^{j\omega})\}$ \\
			\hline
			\multicolumn{3}{|c|}{\textbf{Parseval's Relation for Aperiodic Signals}} \\
			\hline
			\multicolumn{3}{|c|}{$\sum_{n=-\infty}^{+\infty} |x[n]|^2 = \frac{1}{2\pi}\int_{2\pi} |X(e^{j\omega})|^2 \,\dd\omega$} \\
			\hline
		\end{tabular}%
		}
		\vspace{0.5cm}
		\caption{Properties of the Discrete-Time Fourier Transform}
		\label{tab:dtft_properties}
	\end{table}
	
	\begin{table}[H]
		\centering
		\small
		\renewcommand{\arraystretch}{1.2}
		\resizebox{0.95\textwidth}{!}{%
		\begin{tabular}{|>{\raggedright\arraybackslash}p{4cm}|>{\raggedright\arraybackslash}p{4cm}|>{\raggedright\arraybackslash}p{5cm}|}
			\hline
			\textbf{Signal} & \textbf{Fourier Transform} & \textbf{Fourier Series Coefficients (if periodic)} \\
			\hline
			$\sum_{k=\langle N \rangle} a_k e^{jk(2\pi/N)n}$ & $2\pi\sum_{k=-\infty}^{+\infty} a_k \delta(\omega - \frac{2\pi k}{N})$ & $a_k$ \\
			\hline
			$e^{j\omega_0 n}$ & $2\pi\sum_{l=-\infty}^{+\infty} \delta(\omega - \omega_0 - 2\pi l)$ & (a) $\omega_0=\frac{2\pi m}{N}$: $a_k = \begin{cases} 1, & k=m, m\pm N, \dots \\ 0, & \text{otherwise} \end{cases}$ \\
			& & (b) $\omega_0/\pi$ irrational: The signal is aperiodic \\
			\hline
			$\cos(\omega_0 n)$ & $\pi\sum_{l=-\infty}^{+\infty} \{\delta(\omega-\omega_0-2\pi l) + \delta(\omega+\omega_0-2\pi l)\}$ & (a) $\omega_0=\frac{2\pi m}{N}$: $a_k = \begin{cases} \frac{1}{2}, & k=\pm m, \pm m\pm N, \dots \\ 0, & \text{otherwise} \end{cases}$ \\
			& & (b) $\omega_0/\pi$ irrational: The signal is aperiodic \\
			\hline
			$\sin(\omega_0 n)$ & $\frac{\pi}{j}\sum_{l=-\infty}^{+\infty} \{\delta(\omega-\omega_0-2\pi l) - \delta(\omega+\omega_0-2\pi l)\}$ & (a) $\omega_0=\frac{2\pi m}{N}$: $a_k = \begin{cases} \frac{1}{2j}, & k=m, m\pm N, \dots \\ -\frac{1}{2j}, & k=-m, -m\pm N, \dots \\ 0, & \text{otherwise} \end{cases}$ \\
			& & (b) $\omega_0/\pi$ irrational: The signal is aperiodic \\
			\hline
			$x[n]=1$ & $2\pi\sum_{l=-\infty}^{+\infty} \delta(\omega-2\pi l)$ & $a_k = \begin{cases} 1, & k=0, \pm N, \pm 2N, \dots \\ 0, & \text{otherwise} \end{cases}$ (periodic pulse train) \\
			\hline
			Periodic square wave & $2\pi\sum_{k=-\infty}^{+\infty} a_k \delta(\omega - \frac{2\pi k}{N})$ & $a_k=\frac{\sin[(2\pi k/N)(N_1+1/2)]}{N\sin(\pi k/N)}, k\ne 0, \pm N, \dots$ \\
			$x[n]=\begin{cases} 1, & |n|\le N_1 \\ 0, & N_1<|n|\le N/2 \end{cases}$ & & $a_k=\frac{2N_1+1}{N}, k=0, \pm N, \pm 2N, \dots$ \\
			and $x[n+N]=x[n]$ & & \\
			\hline
			$\sum_{k=-\infty}^{+\infty} \delta[n-kN]$ & $\frac{2\pi}{N}\sum_{k=-\infty}^{+\infty} \delta(\omega - \frac{2\pi k}{N})$ & $a_k=\frac{1}{N}$ for all k \\
			\hline
			$a^n u[n], |a|<1$ & $\frac{1}{1-ae^{-j\omega}}$ & \\
			\hline
			$x[n]=\begin{cases} 1, & |n|\le N_1 \\ 0, & |n|>N_1 \end{cases}$ & $\frac{\sin[\omega(N_1+1/2)]}{\sin(\omega/2)}$ & \\
			\hline
			$\frac{\sin(Wn)}{\pi n}, 0<W<\pi$ & $X(e^{j\omega})=\begin{cases} 1, & 0\le|\omega|\le W \\ 0, & W<|\omega|\le\pi \end{cases}$ & \\
			& ($X(e^{j\omega})$ is periodic with period $2\pi$) & \\
			\hline
			$\delta[n]$ & 1 & \\
			\hline
			$u[n]$ & $\frac{1}{1-e^{-j\omega}} + \sum_{k=-\infty}^{+\infty} \pi\delta(\omega-2\pi k)$ & \\
			\hline
			$\delta[n-n_0]$ & $e^{-j\omega n_0}$ & \\
			\hline
			$(n+1)a^n u[n], |a|<1$ & $\frac{1}{(1-ae^{-j\omega})^2}$ & \\
			\hline
			$\frac{(n+r-1)!}{n!(r-1)!}a^n u[n], |a|<1$ & $\frac{1}{(1-ae^{-j\omega})^r}$ & \\
			\hline
		\end{tabular}%
		}
		\vspace{0.5cm}
		\caption{Basic Discrete-Time Fourier Transform Pairs}
		\label{tab:dtft_pairs_comprehensive}
	\end{table}

\newpage
\section*{Summary and Next Lecture}
	\begin{itemize}[noitemsep]
		\item \textbf{Duality:} Interchanges roles of time and frequency
		\item \textbf{Tables as a Tool:} Efficient problem solving by combining basic pairs with properties (shifting, scaling, differentiation, convolution)
		\item \textbf{Next time:} Magnitude and phase of the Fourier transform and their roles in signal structure
	\end{itemize}
	
\end{document}
