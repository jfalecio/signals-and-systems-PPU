\begin{tikzpicture}
	\begin{groupplot}[
		group style={
			group size=2 by 2,
			horizontal sep=2.5cm,
			vertical sep=2.5cm,
			group name=plots,
		},
		% Common styles for all plots
		width=7cm,
		height=5.5cm,
		grid=major,
		title style={font=\bfseries, yshift=-1ex},
		label style={font=\small},
		tick label style={font=\small},
		]
		
		% TOP LEFT: Video frames - Temporal sampling
		\nextgroupplot[
		title={Video: Temporal Sampling},
		xlabel={Time (\si{\milli\second})},
		ylabel={Signal Amplitude},
		xmin=0, xmax=180,
		ymin=0, ymax=7,
		xtick={0, 33.3, 66.7, 100, 133.3, 166.7},
		ytick={1,...,6},
		legend style={at={(0.03,0.97)}, anchor=north west, font=\tiny},
		]
		% Continuous analog signal
		\addplot[blue, smooth, thick, domain=0:180, samples=200, name path=signal] {3.5 + 2.5*sin(360*x/150)};
		\addlegendentry{Analog Signal}
		
		% Sampled points which become the video frames
		\addplot[only marks, red, mark=*, mark size=2.5pt, samples at={33.3, 66.7, 100, 133.3, 166.7}] {3.5 + 2.5*sin(360*x/150)};
		\addlegendentry{Sampled Frames (30 fps)}
		
		% Annotation
		\node[anchor=south, font=\tiny, text width=4cm, align=center] at (axis cs:90,0.1) {Each sample ($\bullet$) is captured as one discrete video frame.};
		
		% TOP RIGHT: Digital image - Spatial sampling
		\nextgroupplot[
		title={Image: Spatial Sampling},
		xlabel={Pixel Column (x)},
		ylabel={Pixel Row (y)},
		xmin=0, xmax=16,
		ymin=0, ymax=9,
		xtick={0,4,8,12,16},
		ytick={0,3,6,9},
		enlargelimits=false,
		view={0}{90}, % Top-down 2D view
		colormap/gray, % Grayscale colormap
		]
		% Efficiently create a pixel grid using a surf plot
		\addplot3[surf, shader=flat, domain=0:15, domain y=0:8, samples=16, samples y=9] 
		{0.5 + 0.5*sin(180*x/15) * cos(180*y/8)};
		
		% Annotation
		\node[anchor=south, font=\tiny, fill=white, fill opacity=0.7, text opacity=1, rounded corners=2pt, inner sep=1pt] at (axis cs:8,0.5) {Each shaded square represents one pixel.};
		
		% BOTTOM LEFT: Heart rate - Physiological sampling
		\nextgroupplot[
		title={Heart Rate: Physiological Sampling},
		xlabel={Time (\si{\second})},
		ylabel={Heart Rate (BPM)},
		xmin=0, xmax=60,
		ymin=55, ymax=90,
		xtick={0,10,20,30,40,50,60},
		ytick={60,70,80,90},
		legend style={at={(0.03,0.97)}, anchor=north west, font=\tiny},
		]
		% Continuous signal (e.g., from an ECG)
		\addplot[blue, smooth, thick, domain=0:60, samples=200] {75 + 8*sin(deg(2*pi*x/20)) + 4*sin(deg(2*pi*x/7))};
		\addlegendentry{Continuous (ECG)}
		
		% CORRECTED sampled points that lie on the curve (e.g., from a wearable)
		\addplot[only marks, red, mark=*, mark size=2.5pt, samples at={0,5,...,60}] {75 + 8*sin(deg(2*pi*x/20)) + 4*sin(deg(2*pi*x/7))};
		\addlegendentry{Sampled (Wearable)}
		
		% BOTTOM RIGHT: Audio - Nyquist Principle
		\nextgroupplot[
		title={Audio: Nyquist Principle (CD Quality)},
		xlabel={Frequency (\si{\kilo\hertz})},
		ylabel={Amplitude (\si{dB})},
		xmin=0, xmax=25,
		ymin=-70, ymax=5,
		xtick={0, 10, 20, 22.05},
		xticklabels={$0$, $10$, $20$, $f_s/2=22.05$},
		ytick={-60, -40, -20, 0},
		legend style={at={(0.97,0.03)}, anchor=south east, font=\tiny},
		]
		% Shade the range of human hearing
		\path[name path=bottom, fill=green!20] (axis cs:0,-70) rectangle (axis cs:20,5);
		\addlegendentry{Human Hearing Range}
		\node[green!50!black, font=\tiny, anchor=south, rotate=90] at (axis cs:19.5, -35) {Audible Frequencies};
		
		% Anti-aliasing filter for 44.1 kHz sampling rate
		\addplot[blue, thick, smooth, domain=0:25, name path=filter] { -3 - 20*log10(1+(x/21.5)^12) };
		\addlegendentry{Anti-Aliasing Filter}
		
		% Nyquist frequency line
		\draw[dashed, red, thick] (axis cs:22.05,-70) -- (axis cs:22.05,5) node[above, sloped, pos=0.8, font=\tiny, black] {Nyquist Frequency};
		
	\end{groupplot}
	
	% --- Main Title and Footnote ---
	\node[font=\large\bfseries] at (plots c1r1.north) [above=1cm] {The Principle of Sampling: From Analog to Digital};
	
	\node[font=\small, text width=15cm, align=center, fill=gray!10, rounded corners, inner sep=2mm] at (plots c1r2.south) [below=1.5cm] {
		\textbf{The Nyquist-Shannon Sampling Theorem:} To perfectly reconstruct an analog signal from its samples, \\ the sampling frequency ($f_s$) must be strictly greater than twice the maximum frequency in the signal ($f_{\text{max}}$). \quad $\boldsymbol{f_s > 2f_{\text{max}}}$
	};
\end{tikzpicture}