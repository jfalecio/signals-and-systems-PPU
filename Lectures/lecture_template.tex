\documentclass[11pt]{article}

\usepackage{amsmath,amssymb,mathtools}
\usepackage[margin=1in]{geometry}
\usepackage{enumitem}
\usepackage{xcolor}
\usepackage{microtype}
\usepackage{graphicx}
\usepackage{tikz,float}
\usepackage{subcaption}
\usepackage{amsthm}
\usepackage{hyperref}

\usetikzlibrary{shapes.geometric, arrows.meta, positioning, calc, decorations.markings}
\tikzset{
	block/.style={rectangle, draw, text width=6em, text centered, rounded corners, minimum height=10mm},
	sum/.style={circle, draw, node distance=1.5cm},
	line/.style={draw, -{Stealth[length=2.5mm, width=1.5mm]}}
}

\usepackage{pgfplots}
\usepgfplotslibrary{groupplots}
\pgfplotsset{compat=1.18}

\pgfplotsset{
	myaxes/.style={
		axis lines=middle,
		grid=both,
		minor grid style={gray!15},
		major grid style={gray!35},
		tick align=outside,
		tick style={black},
		xlabel style={yshift=3pt, anchor=north},
		ylabel style={yshift=-3pt, anchor=south},
		clip=false
	},
	myplotstyle/.style={
		width=14cm,
		height=7cm,
		axis lines=middle,
		axis line style={-Stealth},
		grid=both,
		minor tick num=1,
		major grid style={draw=gray!30},
		minor grid style={draw=gray!15},
		tick label style={font=\small, fill=white, inner sep=1.5pt},
		xlabel={$t$},
		ylabel={$x(t)$},
		xlabel style={anchor=north east, font=\small},
		ylabel style={anchor=south east, font=\small},
		samples=401,
	}
}

\newtheoremstyle{mynote}
{6pt}      % Space above
{6pt}      % Space below
{}         % Body font (normal, not italic)
{}         % Indent amount
{\bfseries}  % Theorem head font
{.}        % Punctuation after theorem head
{.5em}     % Space after theorem head
{}         % Theorem head spec
\theoremstyle{mynote}
\newtheorem{definition}{Definition}
\newtheorem{proposition}{Proposition}
\newtheorem{example}{Example}
\newtheorem{remark}{Remark}
\newtheorem{theorem}{Theorem}
\newtheorem{corollary}{Corollary}

\newcommand{\T}{\mathcal{T}}
\newcommand{\R}{\mathbb{R}}
\newcommand{\Z}{\mathbb{Z}}
\newcommand{\C}{\mathbb{C}}
\newcommand{\conv}{\ast}
\newcommand{\dt}{\mathrm{d}t}
\newcommand{\dd}{\mathrm{d}}
\newcommand{\imp}{\delta}
\newcommand{\sinc}[1]{\frac{\sin(\pi #1)}{\pi #1}}

\begin{document}
	% Reset figure counter for this lecture
	\renewcommand{\thefigure}{X.\arabic{figure}}
	
	% --- TITLE BLOCK ---
	\thispagestyle{empty}
	\noindent
	\begin{tabular*}{\textwidth}{l @{\extracolsep{\fill}} r}
		\textbf{Signals and Systems} & \textbf{Lecture X} \\
		\textit{Ahmed Rabei} & \textit{Fall 2025} \\
	\end{tabular*}
	\hrule
	\vspace{0.4cm}
	\begin{center}
		\Large\textbf{Lecture X: [Lecture Title]}
	\end{center}
	\vspace{0.4cm}
	
\section*{Reference}
	Oppenheim \& Willsky, \textit{Signals and Systems}, Chapter X, Sections X.X--X.X

\section*{Review of Lecture X-1}
	\begin{itemize}[noitemsep]
		\item [Previous lecture key points]
		\item [Previous lecture key points]
		\item [Previous lecture key points]
		\item [Previous lecture key points]
	\end{itemize}

\section*{X.1 [Main Section Title]}
	[Section content]

\subsection*{X.1.1 [Subsection Title]}
	[Subsection content]

\paragraph{[Paragraph Title]:} [Paragraph content]

\begin{definition}
	[Definition content]
\end{definition}

\begin{proposition}
	[Proposition content]
\end{proposition}

\begin{theorem}
	[Theorem content]
\end{theorem}

\begin{example}
	[Example content]
\end{example}

\begin{remark}
	[Remark content]
\end{remark}

\begin{figure}[H]
	\centering
	\input{figures/lecXX/fig_[figure_name].tex}
	\caption{[Figure caption]}
	\label{fig:[figure_label]}
\end{figure}

\begin{figure}[H]
	\centering
	\begin{subfigure}[b]{0.48\textwidth}
		\centering
		\input{figures/lecXX/fig_[subfigure1_name].tex}
		\caption{[Subfigure 1 caption]}
		\label{fig:[subfigure1_label]}
	\end{subfigure}
	\hfill
	\begin{subfigure}[b]{0.48\textwidth}
		\centering
		\input{figures/lecXX/fig_[subfigure2_name].tex}
		\caption{[Subfigure 2 caption]}
		\label{fig:[subfigure2_label]}
	\end{subfigure}
	\caption{[Overall figure caption]}
	\label{fig:[overall_figure_label]}
\end{figure}

\section*{Summary and Next Lecture}
	\begin{itemize}[noitemsep]
		\item [Key takeaway 1]
		\item [Key takeaway 2]
		\item [Key takeaway 3]
		\item [Key takeaway 4]
		\item \textbf{Next time:} [Next lecture preview]
	\end{itemize}

\end{document}