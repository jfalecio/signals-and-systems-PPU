\usepackage{amsmath,amssymb,mathtools}
\usepackage[margin=1in]{geometry}
\usepackage{enumitem}
\usepackage{xcolor}
\usepackage{microtype}
\usepackage{graphicx}
\usepackage{tikz,float}
\usepackage{subcaption}
\usepackage{amsthm}
\usepackage{hyperref}
\usepackage{array}
\usepackage{pgfplots}

\usetikzlibrary{shapes.geometric, arrows.meta, positioning, calc, decorations.markings}
\tikzset{
	block/.style={rectangle, draw, text width=6em, text centered, rounded corners, minimum height=10mm},
	sum/.style={circle, draw, node distance=1.5cm},
	line/.style={draw, -{Stealth[length=2.5mm, width=1.5mm]}}
}

\usepgfplotslibrary{groupplots}
\pgfplotsset{compat=1.18}

\pgfplotsset{
	myaxes/.style={
		axis lines=middle,
		axis line style={-latex},
		grid=major,
		grid style={gray!15},
		minor grid style={gray!35},
		xlabel style={at={(ticklabel* cs:1)}, anchor=north west},
		ylabel style={at={(ticklabel* cs:1)}, anchor=south east},
		every axis plot/.append style={thick}
	},
	myplotstyle/.style={
		width=14cm,
		height=7cm,
		axis lines=middle,
		axis line style={-Stealth},
		grid=both,
		minor tick num=1,
		major grid style={draw=gray!30},
		minor grid style={draw=gray!15},
		tick label style={font=\small, fill=white, inner sep=1.5pt},
		xlabel={$t$},
		ylabel={$x(t)$},
		xlabel style={anchor=north east, font=\small},
		ylabel style={anchor=south east, font=\small},
		samples=401,
	}
}

\newtheoremstyle{mynote}
{6pt}      % Space above
{6pt}      % Space below
{}          % Body font (normal, not italic)
{}          % Indent amount
{\bfseries} % Theorem head font
{.}         % Punctuation after theorem head
{.5em}      % Space after theorem head
{}          % Theorem head spec
\theoremstyle{mynote}
\newtheorem{definition}{Definition}
\newtheorem{proposition}{Proposition}
\newtheorem{example}{Example}
\newtheorem{remark}{Remark}
\newtheorem{theorem}{Theorem}
\newtheorem{corollary}{Corollary}

\newcommand{\T}{\mathcal{T}}
\newcommand{\R}{\mathbb{R}}
\newcommand{\Z}{\mathbb{Z}}
\newcommand{\C}{\mathbb{C}}
\newcommand{\conv}{\ast}
\newcommand{\dt}{\,\dd t}
\newcommand{\dd}{\mathrm{d}}
\newcommand{\imp}{\delta}
\newcommand{\sinc}[1]{\frac{\sin(\pi #1)}{\pi #1}}


\DeclareMathOperator{\rect}{rect}
\DeclareMathOperator{\Ev}{Ev}
\DeclareMathOperator{\Od}{Od}
\DeclareMathOperator{\sgn}{sgn}
\DeclareMathOperator{\step}{u}
\DeclareMathOperator{\tri}{tri}
